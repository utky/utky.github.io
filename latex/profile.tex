\documentclass[9pt]{jsarticle}
\begin{document}

\begin{center}
  {\Large 職務経歴書}
\end{center}

\begin{flushleft}
  \today
\end{flushleft}

\begin{flushright}
  今村 豊
\end{flushright}

\section{主要スキル}

ネットワークの制御・監視システム開発に3案件、合計5年間従事。
直近3年間の案件ではMPLS網を構成する200ノードの監視とNETCONFによる制御を行うシステムを構築。
仕様策定・設計・実装・試験・導入まで担当する。
現在のスキルセットの多くはこの案件での経験に基づく。

スキルについてのポイントは下記の通り。

\begin{enumerate}
  \item MPLSネットワークの制御・監視ノウハウ
  \item 自動化技術を活用した開発環境の改善
  \item 関数型プログラミングの積極的導入
\end{enumerate}

\begin{description}
    \item[プログラミング言語] Haskell, Java, Python, JavaScript, Clojure
    \item[OS] CentOS, Ubuntu, Debian, Junos, Cisco IOS, Arista EOS
    \item[ミドルウェア] Docker, PostgreSQL, MySQL, Elasticsearch
    \item[ネットワーク技術] NETCONF, OSPF, BGP, MPLS, OpenFlow, NetFlow
\end{description}

\section{勤務履歴}

\begin{itemize}
  \item 2008年  4月 ソフトウエア興業株式会社 入社
  \item 2010年 12月 ソフトウエア興業株式会社 退社
  \item 2011年  1月 株式会社ユニスティ 入社 
\end{itemize}

\section{実務経験経歴}

\subsection{1社目}

ソフトウエア興業株式会社 (2008年4月 〜 2010年12月)

\subsubsection{MPLS伝送網運用システム開発}

\begin{description}
    \item[プロダクト] MPLSを用いた光伝送システムを制御するWebシステム。専用の伝送装置へと命令を発行し回線制御を行う。
    \item[業種] 通信
    \item[期間] 2009年6月〜2010年12月
    \item[担当工程] 技術調査, 基本設計, 詳細設計, 製造, 単体試験, 結合試験, 総合試験
    \item[役割] プログラマ
    \item[プロジェクト規模] 20人程度
    \item[使用技術] Java, Apache Tomcat, Java Server Faces, MySQL, MPLS, SNMP
    \item[貢献] 仕様策定チームと議論をしながら仕様バグを摘出
\end{description}

\subsection{2社目}

株式会社ユニスティ (2011年1月 〜 現在)

\subsubsection{案件管理Webシステム開発}

\begin{description}
    \item[プロダクト] 顧客企業内での開発案件ワークフローを電子化。他システムと連携し自動デプロイメントも行う。
    \item[業種] Web
    \item[期間] 2011年1月〜2013年2月
    \item[担当工程] 技術調査, 計画, 見積もり, 要件定義, 基本設計, 詳細設計, 製造, 単体試験, 結合試験, 総合試験
    \item[役割] サブリーダ
    \item[プロジェクト規模] 4人
    \item[使用技術] Java, Java Server Faces, WebLogic, Java Messsage Service, Oracle Database
    \item[貢献] UIフレームワークのバージョンアップとデータベース移行を主導
\end{description}

\subsubsection{MPLS映像伝送網運用システム開発}

\begin{description}
    \item[プロダクト] MPLSベースの映像伝送網の制御および監視を行う。スケジューラによる時間ベースの回線制御を提供。
    \item[業種] 通信
    \item[期間] 2013年7月〜2016年11月
    \item[担当工程] 技術調査, 計画, 要件定義, 基本設計, 詳細設計, 製造, 単体試験, 結合試験
    \item[役割] サブリーダ
    \item[プロジェクト規模] 3人
    \item[使用技術] MPLS, NETCONF, SNMP, Python, Zabbix, PostgreSQL, React.js
    \item[貢献] CI環境の整備などの開発環境を改善
\end{description}

\subsubsection{クラウド向けSDNコントローラ開発}

\begin{description}
    \item[プロダクト] OpenStack用仮想ネットワークを提供するSDNコントローラ
    \item[業種] 通信
    \item[期間] 2016年12月〜現在
    \item[担当工程] 受入試験
    \item[役割] プロジェクトメンバ
    \item[プロジェクト規模] 4人
    \item[使用技術] NETCONF, SNMP, Python, MySQL
    \item[貢献] 貢献中...
\end{description}


\section{学歴}

2008年3月 神奈川大学 外国語学部 英語英文学科 卒業

\section{保有資格}

\begin{itemize}
  \item 基本情報技術者(旧第二種情報処理技術者) (2008年取得)
  \item ソフトウェア開発技術者(旧第一種情報処理技術者) (2008年取得)
  \item 情報セキュリティスペシャリスト (2015年取得)
\end{itemize}

\section{自己PR}

運用システムの開発を通じて、開発環境の高度化に取り組みました。
Dockerを活用してオンプレミスにCI環境を構築し、複数の実行環境を分離してテストすることが出来ました。
また、Ansibleを活用して他チームの作業ルーティンやアプリケーションデプロイの手順を自動化することも出来ました。
現在は他のチームも含めてタスクの消化状況を見える化するための仕組みづくりに取り組んでいます。

プライベートでプログラムを書く際は、純粋関数型プログラミング言語Haskellを使用しています。
Haskellがもたらす堅牢さやバグの少なさといった恩恵を、ネットワーク系のサービス開発にも導入するべく、
多くの開発者にHaskellの素晴らしさを伝え、Haskell開発の足場を作っていきたいと思っています。

\begin{flushright}
  以上
\end{flushright}

\end{document}
